\paragraph{Dimostrazione dell'integrale: }
In generale vale che: 
\[[1]: \ \  I_n = \int_0^\infty \frac{x^n}{e^x-1}dx = \int_0^\infty x^n\sum_{i=1}^{\infty} e^{-ix}dx \]
\textit{Dimostriamolo}: La sommatoria introdotta è una serie geometrica caratterizzata dall'andamento
\[\sum_{n=0}^{\infty}q^n = 1 + q + q^2 + q^3 + ... + q^n + ...\])\\
Se consideriamo la serie geometrica $e^{-ix}$, poiché $n = -ix$ è considerata una serie parte delle serie che hanno $q<1$ perchè:\\
\begin{cent}
se $x>0$ allora l'esponenziale $e^{-x} \in (0,1) \to 0<e^{-x}<1$
\end{cent}
\\Le serie geometriche di ragione q<1 si risolvono: 
\[ \sum_{n=0}^{\infty} q^n = \frac{1}{1-q}\]
Applicando questa proprietà abbiamo
\[ \sum_{n=0}^{\infty} e^{-x} = \frac{1}{1-e^{-x}} = \frac{1}{1-\frac{1}{e^x}} = \frac{e^x}{e^x-1}\]
E dunque
\[ [2]: \ \ \sum_{n=1}^{\infty} e^{-x} = \sum_{n=0}^{\infty} e^{-x} -1  = \frac{e^x}{e^x-1} -1 =  \frac{e^x-e^x + 1}{e^x -1} = \frac{1}{e^x -1} \]
Una volta dimostrato la [2], vale la [1]. Poiché l'integrale di una somma è la somma degli integrali vale dunque 
\[ \I_n = \sum_{i=1}^{\infty} \int_0^\infty x^n e^{-ix}dx \]
